% \vspace{0.15cm}
\cvsection{Projects in Physics}
\textbf{Field of interest:} super-heavy elements, quasimolecules, QED

\divider

\cvproject{Electronic structure of heavy few-electron diatomic quasimolecules, G-RISC}
\begin{itemize}
    \item Research on the configuration interaction method in the diatomic quasimolecules
    \item Development of the package to perform the electronic correlation calculation on the highest accuracy up-to-date.
\end{itemize}

\cvproject{Energy Spectra of Diatomic Quasimolecules}
\begin{itemize}
    \item Development of the numerical program for the electronic spectra calculation of the diatomic quasimolecules based on A-DKB B-Splines.
    \item Research and development of the package for the interelectronic interaction energy calculation.
    \item Optimization of the resource consumption by program up to 40\% in RAM.
\end{itemize}

\cvproject{Super-heavy nuclei and atoms: mass limit of nuclei and boundary of the periodic table}
\begin{itemize}
    \item Property calculation of the super-heavy molecules and atoms such as dipole moment, polarizability, optimal geometry, etc.
    \item Calculation via coupled-cluster approach implemented in DIRAC.
\end{itemize}

\cvsection{Projects in ML}
\textbf{Field of interest:} generative models in CV, multimodal models, image/video editing via neural networks, DL applications to physics problems

\divider

\cvproject{Generative Makeup}
\begin{itemize}
    \item Research and development of generative-adversarial network (GAN) application to makeup generation
    \item Controllable generation of makeup images using latent-search approaches (GANSpace, StyleCLIP, etc.)
    \item Makeup transfer using neural-network-based methods
\end{itemize}

\cvproject{What? Where? When?}
\begin{itemize}
    \item Development of Telegram-bot for "What? Where? When?" game with generative questions similar to real one with GPT-based model
    \item Theme modelling of the questions from "What? Where? When?" game with RuBert embeddings
\end{itemize}

\cvproject{RL in Quantum Computing}
\begin{itemize}
    \item Research and development of RL algorithms (Maskable PPO) for optimization of quantum circuits within ZX-calculus framework
\end{itemize}

\cvproject{Photonic crystal optimization}
\begin{itemize}
    \item Development of the photonic crystal optimization algorithm via 1) VAE approach to images of photonic crystal structures and 2) learnable representation within k-space approach
\end{itemize}

\cvproject{}