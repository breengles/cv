%%%%%%%%%%%%%%%%%
% This is an example CV created using altacv.cls (v1.1.5, 1 December 2018) written by
% LianTze Lim (liantze@gmail.com), based on the
% Cv created by BusinessInsider at http://www.businessinsider.my/a-sample-resume-for-marissa-mayer-2016-7/?r=US&IR=T
%
%% It may be distributed and/or modified under the
%% conditions of the LaTeX Project Public License, either version 1.3
%% of this license or (at your option) any later version.
%% The latest version of this license is in
%%    http://www.latex-project.org/lppl.txt
%% and version 1.3 or later is part of all distributions of LaTeX
%% version 2003/12/01 or later.
%%%%%%%%%%%%%%%%

%% If you are using \orcid or academicons
%% icons, make sure you have the academicons
%% option here, and compile with XeLaTeX
%% or LuaLaTeX.
% \documentclass[10pt,a4paper,academicons]{altacv}

%% Use the "normalphoto" option if you want a normal photo instead of cropped to a circle
% \documentclass[10pt,a4paper,normalphoto]{altacv}

\documentclass[10pt,a4paper,ragged2e, academicons]{altacv}
\usepackage{hyperref}
% American style citations in bibliography
\renewbibmacro*{journal+issuetitle}{%
\usebibmacro{journal}%
\setunit*{\addspace}%
\iffieldundef{series}
{}
{\newunit
\printfield{series}%
\setunit{\addspace}}%
\usebibmacro{volume+number+eid}%
\setunit{\bibpagespunct}%
\printfield{pages}%
\setunit{\addspace}%
\usebibmacro{issue+date}%
\setunit{\space}%
\usebibmacro{issue}%
\newunit}

\renewbibmacro*{note+pages}{%
\printfield{note}%
\newunit}

\DeclareFieldFormat[article,periodical]{pages}{#1}
\renewbibmacro{in:}{} % Remove In:
\DeclareFieldFormat[article]{volume}{\textbf{#1}\space} % Bold volume
\DeclareFieldFormat[article]{journaltitle}{#1} % Journal title is printed as normal text
% \DeclareFieldFormat[article,incollection,unpublished]{title}{\textit{#1}} % No quotes for article \textit{titles}
\DeclareFieldFormat[article,incollection,unpublished]{title}{} % Remove article titles in bibliography
\DeclareFieldFormat{pages}{#1} % Remove p./pp.

%% AltaCV uses the fontawesome and academicon fonts
%% and packages.
%% See texdoc.net/pkg/fontawecome and http://texdoc.net/pkg/academicons for full list of symbols. You MUST compile with XeLaTeX or LuaLaTeX if you want to use academicons.

% Change the page layout if you need to
\geometry{left=2cm,right=10cm,marginparwidth=6.8cm,marginparsep=1.2cm,top=1.25cm,bottom=1.25cm}

% Change the font if you want to, depending on whether
% you're using pdflatex or xelatex/lualatex
\ifxetexorluatex
  % If using xelatex or lualatex:
  \setmainfont{Carlito}
\else
  % If using pdflatex:
  \usepackage[utf8]{inputenc}
  \usepackage[T1]{fontenc}
  \usepackage[default]{lato}
\fi

% Change the colours if you want to
\definecolor{VividPurple}{HTML}{000000}
\definecolor{SlateGrey}{HTML}{2E2E2E}
\definecolor{LightGrey}{HTML}{2E2E2E}
\colorlet{heading}{VividPurple}
\colorlet{accent}{VividPurple}
\colorlet{emphasis}{SlateGrey}
\colorlet{body}{LightGrey}

% Change the bullets for itemize and rating marker
% for \cvskill if you want to
\renewcommand{\itemmarker}{{\small\textbullet}}
\renewcommand{\ratingmarker}{\faCircle}

%% sample.bib contains your publications
\addbibresource{sample.bib}

\begin{document}
\name{Artem Kotov}
\tagline{PhD Student @ St. Petersburg State University | Junior ML Engineer}
% Cropped to square from https://en.wikipedia.org/wiki/Marissa_Mayer#/media/File:Marissa_Mayer_May_2014_(cropped).jpg, CC-BY 2.0
\personalinfo{%
    \location{St. Petersburg, Russia}
    % You can add your own with \printinfo{symbol}{detail}
    \email{\href{mailto:artem.a.kotov@gmail.com}{artem.a.kotov@gmail.com}}
    \telegram{\href{https://t.me/breengles}{t.me/breengles}}
    %  \mailaddress{Address, Street, 00000 County}
    %  \homepage{marissamayr.tumblr.com/}
    %  \twitter{@marissamayer}
    %  \linkedin{linkedin.com/in/ronak-dedhiya}
    \github{\href{https://github.com/breengles}{github.com/breengles}} % I'm just making this up though.
    % \orcid{orcid.org/0000-0002-4629-138X}
    \orcid{\href{https://orcid.org/
            0000-0002-4629-138X}{0000-0002-4629-138X}}
}

%% Make the header extend all the way to the right, if you want.
\begin{fullwidth}
    \makecvheader
\end{fullwidth}

%% Depending on your tastes, you may want to make fonts of itemize environments slightly smaller
\AtBeginEnvironment{itemize}{\small}

%% Provide the file name containing the sidebar contents as an optional parameter to \cvsection.
%% You can always just use \marginpar{...} if you do
%% not need to align the top of the contents to any
%% \cvsection title in the "main" bar.
\cvsection[page1sidebar]{Experience}

\cvevent{Engineer Researcher}{Quantum Mechanics Lab @ St. Petersburg State University}{Aug 2018 -- Present}{St. Petersburg, Russia}
\begin{itemize}
    \item Research and development of numerical algorithms for a relativistic spectrum calculation of the diatomic quasimolecules
    \item Performing chemical property calculation of the super-heavy elements and molecules
\end{itemize}

\divider

\cvevent{Teaching}{St. Petersburg State University}{Feb 2018 -- Jun 2018}{St. Petersburg, Russia}
\begin{itemize}
    \item Theoretical and practical course on the introduction to the quantum mechanics for college students
    \item Practical course on the quantum mechanics for BSc. students
\end{itemize}


\cvsection{Education / Courses}
\cvevent{PhD student in Physics}{St. Petersburg State University}{Sep 2020 -- Present}{St. Petersburg, Russia}
\cvevent{MSc. in Physics}{St. Petersburg State University}{Sep 2018 -- June 2020}{}
\cvevent{BSc. in Physics}{St. Petersburg State University}{Sep 2014 -- June 2018}{}

\divider

\cvevent{Master student in Machine Learning}{Higher School of Economics}{Sep 2020 -- Present}{St. Petersburg, Russia}

% \cvsection{A Day of My Life}
% Adapted from @Jake's answer from http://tex.stackexchange.com/a/82729/226
% \wheelchart{outer radius}{inner radius}{
% comma-separated list of value/text width/color/detail}
% Some ad-hoc tweaking to adjust the labels so that they don't overlap
% \wheelchart{1.5cm}{0.5cm}{%
%   10/10em/accent!30/Sleeping \& dreaming about work,
%   25/9em/accent!60/Public resolving issues with Yahoo!\ investors,
%   5/13em/accent!10/\footnotesize\\[1ex]New York \& San Francisco Ballet Jawbone board member,
%   20/15em/accent!40/Spending time with family,
%   5/8em/accent!20/\footnotesize Business development for Yahoo!\ after the Verizon acquisition,
%   30/9em/accent/Showing Yahoo!\ employees that their work has meaning,
%   5/8em/accent!20/Baking cupcakes
% }

% \clearpage

% \begin{fullwidth}
\cvsection[page2sidebar]{Publications}

\nocite{*}

% \printbibliography[heading=pubtype,title={\printinfo{\faBook}{Books}},type=book]

% \divider

% \printbibliography[heading=pubtype,title={{Journal Articles}}, type=article]
\vspace{-0.5cm}
\subsection*{Journal articles}
\begin{itemize}
    \item A. A. Kotov \textit{et al.} Atoms \textbf{9}, 44 (2021)
    \item A. A. Kotov \textit{et al.} X-Ray Spectrometry \textbf{49}, 110 (2020)
\end{itemize}

\divider

\printbibliography[heading=pubtype,title={{Conferences}}, type=inproceedings]
% \end{fullwidth}
% %% If the NEXT page doesn't start with a \cvsection but you'd
% %% still like to add a sidebar, then use this command on THIS
% %% page to add it. The optional argument lets you pull up the
% %% sidebar a bit so that it looks aligned with the top of the
% %% main column.
% % \addnextpagesidebar[-1ex]{page3sidebar}


\end{document}
